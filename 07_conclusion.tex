\pagestyle{plain}

First goal of the project was to prepare TEM samples which was done using copper grid and potential errors caused by this part of project were evaluated in chapter \ref{measurment_errors} in Discussion. Five samples of nanoparticles of four different sizes were prepared and two samples of nanorods with different peak absorption wavelength were prepared.

Second goal of the project was to get TEM images of nanoparticles and nanorods. Several image of each sample were acquired by TEM by microscope technician. Examples of TEM images are showed in chapter \ref{TEM_images} in Results. Possible measurnemt errors were evaluated in chapter \ref{image_errors_errors} in Discussion.

Third goal of the project was to sugget methods for image segmentatin of these TEM images of nanoparticles and nanorods. Two segmentation methods were chosen on supervisor's recommendation. First method is called watershed transform and it was main method, other method is called hough transform and it was used just for overlapping particles. Both segmentation methods are descibed in chapter \ref{segmentation} in Methods.

Forth goal of the project was to create Python program for automatic analysis of nanoparticles and nanorods from TEM images. Code for this program can be seed in Attachement. Example outputs of program can be seen in chapter \ref{outputs} in Results and were evaluateed in Discussion. Program features were shortly descibed in chapter \ref{program} in Results.

Fifth goal of the project was to create GUI for program.