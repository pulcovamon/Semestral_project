\documentclass[a4paper,12pt]{article}   % Definice - velikost dokumentu, základní velikost písma, typ
\usepackage[a4paper, top=2.5cm, left=3.5cm, right=2.5cm, bottom=2.5cm]{geometry}		% nastavení okrajů

%% Přidání balíčků podporujících různé funkcionality - ZÁKLADNÍ
\usepackage{amsmath,float}				% balíček pro matiku
\usepackage[dvipsnames]{xcolor}
\usepackage{color}

\usepackage{float}						% plovoucí prostředí
\usepackage[utf8]{inputenc}			    % kódování	
\usepackage[czech]{babel}				% čeština
\usepackage{enumerate}      			% seznamy 
\usepackage{amsfonts}      				% množiny Z,R,N dvojitě
\usepackage{amssymb}      				% znaky úhlu a tak
\usepackage[pdftex]{graphicx}
\usepackage{setspace}
\usepackage{multicol}					% tabulka, slučování sloupců
\usepackage{multirow}                   % tabulka, slučování řádků
\usepackage{fancyhdr}					% záhlaví a zápatí stránky
\usepackage{chngcntr}                   % číslování (číslování rovnic, obrázků dle kapitol)
\usepackage{array}                      % rozšíření práce s tabulkami
\usepackage{helvet}                     % předefinuje \sfdefault to uhv (pro úvodní stránku)
\usepackage[flushleft]{threeparttable}  % prostředí pro tabulky - přidává vysvětlující poznámky pod tabulku

% Následující balíčky řeší citace v dokumentu. 
\usepackage{csquotes}                       
\usepackage[style=iso-numeric]{biblatex}     
\addbibresource{reference_state_of_art.bib}             % zdrojový soubor s citacemi

%% Přidání balíčků podporujících různé funkcionality - VOLITELNÉ, DOPLŇKOVÉ
% Existuje celá řada dalších
\usepackage[]{algorithm2e}				    % balíček pro pseudokód
%\usepackage{ifthen}                        % pro algoritmy if else
%\usepackage{paralist}                      % Rozšířená možnost pro seznamy. Větší škála a možnosti jak seznamy dělat autoamticky. 
%\usepackage{fontspec}                      % specifické fonty
%\usepackage{icomma}      				    % není mezera za desetinnou čárkou
% \usepackage[titletoc,title]{appendix}     % automatické vytvoření příloh

%% Přidání speciálních příkazů 
\newcommand{\at}{\makeatletter @\makeatother}           % Vytiskne zavináč - \at
\newcommand{\degree}[1][]{\ensuremath{{#1}^\circ}}      % Vytiskne stupeň - \degree
\newcolumntype{C}[1]{>{\centering\let\newline\\\arraybackslash\hspace{0pt}}m{#1}}
% zarovnání v tabulce, vycentrování, potřebuje \usepackage{array}


%% Ostatní definice a nastavení
\clubpenalty 10000		% penalizace sirotků, sirotek: poslední řádek odstavce je na nové stránce
\widowpenalty 10000		% penalizace vdov, vdova: první řádek nového odstavce je na konci stránky



\DeclareGraphicsExtensions{.pdf,.png,.jpg}	    % nahrávání obrázků, rošíření
\graphicspath{{obrazky/}} 				        % umístění obrázků

\counterwithin{figure}{section}         % číslování obrázků dle sekcí/kapitol
\counterwithin{table}{section}          % číslování tabulek dle sekcí/kapitol
\numberwithin{equation}{section}        % číslování rovnic dle sekcí/kapitol
\usepackage[figurename=Fig.]{caption}


\setlength{\parskip}{6pt}                   % odsazení mezi odstavci
\setlength{\parindent}{0.75cm}              % odsazení odstavců od okraje
\renewcommand{\baselinestretch}{1.20}	    % řádkování 1,2 - odpovídá pevnému řádkování 17 bodů

\usepackage{tocloft}                    % Nastaví tučné zvýraznění sekcí v obsahu
\renewcommand{\cftsecleader}{\cftdotfill{\cftdotsep}}   % přidá vodící linku do obsahu u sekcí


%% Ostatní neimplementované, pouze návod jak případně doplnit
% Vytvořit seznam použitých algoritmů a přejmenování názvu objektu na Algoritmus.
% \usepackage[algosection]{algorithm2e}
% \SetAlgorithmName{Algoritmus}{algorithm}{Seznam algoritmů}
   

%% Deklarace názvů - přepsat dle autora
\newcommand{\autor}{Monika Pulcová}
\newcommand{\vedouci}{doc. Ing. Vladimíra Petráková PhD.}
\newcommand{\nazevENG}{Automatical determination of size and shape of nanoparticles for biomedical applications}
\newcommand{\nazev}{Automatické určení tvaru a velikosti nanočástic pro biomedicínské aplikace}
\newcommand{\typ}{Semestrální projekt 2}
\newcommand{\rok}{2023}
% Pro program a obor jsou instrukce zde: http://www.fbmi.cvut.cz/fakulta/uredni-deska 
% Nově akreditované mají pouze studijní programy
\newcommand{\program}{Biomedicínská technika}

% názvy obrázků
%\newcommand{\}{}

%% Vlastní začátek dokumentu
\begin{document}

	\begin{titlepage}
 		\begin{center}
 		\begin{figure}[!h]
			\centering
 			\includegraphics[width=0.2\textwidth]{symbol_cvut_konturova_verze}
 		\end{figure}
 		\textsf{\large{\textbf{ČESKÉ VYSOKÉ UČENÍ TECHNICKÉ V PRAZE}}}\\
 		%\vspace{0pt}   		
         {\color{NavyBlue}\makebox[\linewidth]{\rule{\textwidth}{0.4mm}}}
         %{\color{NavyBlue}\hrule }  	    \vspace{6pt}
 		\textsf{\normalsize{\textbf{FAKULTA BIOMEDICÍNSKÉHO INŽENÝRSVÍ}}}\\
		\textsf{\textbf{Katedra biomedicínské techniky}}\\	
		
		\vfill
 		
		\textsf{\Large{\textbf{\nazev}}}\\
	    \vspace{24pt}
		\textsf{\Large{\textbf{\nazevENG}}}\\
		\vspace{24pt}
		\textsf{\typ}\\ 
		\vfill
		\end{center}
		\textsf{Studijní program: \program}\\
		\textsf{Vedoucí práce: \vedouci}\\
				
		\begin{center}
		\textsf{\textbf{\autor}} \\ [0.5cm] 
		
		{\color{NavyBlue}\makebox[\linewidth]{\rule{\textwidth}{0.4mm}}} 
 		
		\textsf{\textbf{Kladno \rok}}
		\end{center}

	\end{titlepage}
    \clearpage

    \null\vfill
	\section*{Prohlášení}
	% \hspace ruší odsazení odstavce - v šabloně u prohlášení odstavce odsazené nejsou. 
    Prohlašuji, že jsem semestrální projekt 2 s názvem \uv{\nazev} vypracovala samostatně a~použila k~tomu úplný výčet citací použitých pramenů, které uvádím v seznamu přiloženém k~semestrálnímu projektu 2.

    \hspace{-0.75cm}Nemám závažný důvod proti užití tohoto školního díla ve smyslu \S 60 Zákona č.121/2000~Sb., o~právu autorském, o~právech souvisejících s právem autorským a~o~změně některých zákonů (autorský zákon), ve znění pozdějších předpisů. 
    
    \vspace{1em}
    
    \hspace{-0.75cm}V Kladně dne \ldots \ldots \ldots \hfill \ldots \ldots \ldots \ldots \ldots \ldots \ldots \ldots \ldots \ldots

    \hspace{10cm} \textbf{\autor}

	\clearpage
	
	\null\vfill
	\section*{Acknowledgement}
	I would like to thank my supervisor \vedouci for all her help and time, I would also like to thank Ing. Miroslav Hekrdla PhD. for very significant help with segmentation methods, Python code and text of project, this support was totally crucial. I would also like to thank Ing. Niklas Hansen for support with every laboratory work and for TEM images.

    \clearpage	

    	\null\vfill	
	\section*{ABSTRACT}
        \subsection*{\nazevENG:}
		 
       Nanotechnology is widely used in biomedical research. For their usage it is important to know sizes and shapes of particles. These parameters can be measured by transmission electron microscopy, but analysis of images from transmission electron microscope (TEM) is complicated and tedious. It would be easier if method for automatic evaluation of these parameters would exist. The goal of this project is to create program for automatic determination of sizes and shape of nanoparticles and nanorods from TEM images.

       TEM samples of nanoparticles and nanorods of various sizes were used and TEM images were acquired. Image analysis methods were used for image adjustement and two image segmentation methods were used. First method is watershed transform and second method is circle hough transform.

       Python program was created and tested in various images of nanoparticles and nanorods. Sizes and shapes of nanoparticles and nanords were calculated and histograms of sizes were plotted.

       Possible errors may be caused by wet sample or by low quality of TEM images.

       The goals of the project were completed and evaluated.
    
	\subsection*{Key words}
		nanoparticles, transmission electron microscopy, image segmentation, watershed transform, circle hough transform
	\clearpage

	
	\null\vfill	
	\section*{ABSTRAKT}
        \subsection*{\nazev:}
    Nanotechnologie je široce využívána v biomedicínském výzkumu. Pro její využití je nezbytné znát velikosti a tvary použitých nanočástic. Tyto parametry se běžně měří pomocí transmisní elektronové miskroskopie, ale analýza dat z tohoto mikroskopu je náročná a zdlouhavá. Analýzu by zjednodušila metoda pro automatické určení těchto parametrů. Cílem tohoto projektu je vytvořit program pro automatické určení veliskoti a tvaru nanočástic ze snímků ze elektronového mikroskopu.

    Pro měření bylo využito několik vzorků nanočástic a nanotyčinek různých velikostí, tyto vzorky jsou určeny pro transmisní elektronovou mikroskopii. Bylo pořízeno několik mikroskopických snímků každého vzorku. Pro analýzu snímků byly využity metody analýzy obrazu, primárně segmentace obrazu. Byly využity dvě metody pro segmentaci a to watershed transfomace a hough transformace.

    Byl vytvořen program v jazyce Python pro automatickou analýzu mikroskopických snímků nanočástic, tento program byl testován na snímcích nanočástic různých velikostí a tvarů. Výstupy programu jsou vypočtené velikosti a tvary nanočástic a histogramy velikostí.

    Chyby měření mohly být způsobeny nedostatečně vysušeným vzorkem nebo nízkou kvalitou mikroskopických snímků.

    Cíle byly splněny a zhodnoceny.
     
	\subsection*{Klíčová slova}
		nanočástice, transmisní elektronová mikroskopie, segmentace obrazu, watershed transformace, kruhová hough transformace
	\clearpage
		
	
    \pagestyle{plain}	% Číslování stránek začíná odsud
	
	\tableofcontents			% Vloží obsah

	\clearpage

	\section*{Seznam symbolů a zkratek} %sekce = nadpis, s * neni v obsahu
	\addcontentsline{toc}{section}{Seznam symbolů zkratek}
	\pagestyle{plain}

\subsection{List of symbols}

\begin{center}
\begin{tabular}{c c c c} 
 \hline
 \textbf{Symbol} & \textbf{Unit} & \textbf{Meaning} \\
 \hline\hline
 cx, cy & px & Coordinates of center of circle in image \\ 
 \hline
 r & px & Circle radius in image \\
 \hline
\end{tabular}
\end{center}

\subsection{List of shortcuts}

\begin{center}
\begin{tabular}{c c c c} 
 \hline
 \textbf{Shortcut} & \textbf{Meaning} \\
 \hline\hline
 NP & Nanoparticle \\ 
 \hline
 AuNP & Gold nanoparticle \\
 \hline
 NR & Nanorod \\
 \hline
 GNR & Gold nanorod \\
 \hline
 TEM & Transmission electron microscopy \\
 \hline
 HT & Hougt transform \\
 \hline
 CHT & circle hough transform \\
\hline
 ROI & Region of interestt \\
 \hline
\end{tabular}
\end{center}
	\clearpage
		
	    %\addcontentsline{toc}{section}{Seznam tabulek}
		%\listoftables 		% seznam tabulek
	
		%\clearpage 			% kones stránky a odskok na další
		
		\addcontentsline{toc}{section}{List of figures}
		\listoffigures 		% seznam obrázků
		\clearpage
	
		\addcontentsline{toc}{section}{List of algorithms}
		\listofalgorithms
		\clearpage
  
        \section{Introduction}
        \pagestyle{plain}

Nanoparticles and nanorods are widely used in biomedical research. For their usage it’s essential to know the size and shape of these particles in solution. The most used method for this estimation is transmission electron microscopy (TEM), which is very well known method, but data evaluation is very complicated and includes several image processing methods. The biggest part of image processing of TEM images is image segmentation. There are plenty of segmentation algorithms, but each data is unique and requires totally different approach. Other huge problem is that nanoparticles and nanorods are often overlapping, so the perfect method should divide them well and also take in account overlapping areas. TEM images also contain lot of noise, particles may have inhomogeneous background, even particles may not be of homogenous intensity. Simply there are so many sizes and shapes of nanoparticles or nanorods, that it is challenging to create algorithm which can automatically analyse TEM images of nanoparticles.
        \clearpage
        
        \section{State of art}
        \input{02_state_of_art}
        \clearpage
        
        \section{Goals}
        \pagestyle{plain}

The goal of the project is to create program for automatic determination of sizes and shape of nanoparticles and nanorods from TEM images. Particular goals are:

\begin{itemize}
  \item Prepare TEM samples with nanoparticles and nanorods.
  \item Get TEM images of nanoparticles and nanorods.
  \item Suggest proper segmentation method for TEM images.
  \item Create Python program for automatic analysis.
  \item Make GUI for the program.
\end{itemize}
        \clearpage
        
        \section{Methods}
        section{methods}

\pagestyle{plain}

subsection{Measurement}

subsection{Image analysis}

Another step is creating program in image analysis, it was writen in Python.
There were used several Python libraries, most of segmentation methods were
implemented using Scikit image library, there were also used opencv for Python
and Scipy ndimage for some functions, these libraries are using Numpy, which
is library used for work with martices. There was also used Matplotlib Pyplot
library for plotting and saving images.

First step after loading TEM image is convering it into grayscale due it could
be filtrated. TEM images can be of different scales and size and number of particles
in image depends on it. Median filter is used for denoising image. Filtrating works
of principle of convolution and size and appearance of convolution mask determine type
of filter. Median filter belongs to nonlinear filters which means that output is not
linear function of input. Median filter simply counts median value from neighborhood
of certain pixel for every pixel in image. Advantage of this filter is that it
preserv edges and removes noise. Size of kernel or pixel neighborhood depends on
scale of input image and also on type of particles because nanords are usually
smaller than nanoparticles so it has to be blured less.

Then grayscale image has to be transform into binary image. It is usually done
by finding appropriate threshold and pixels with value below threshold are 
transformed to zero and pixels with value above threshold are transformed into
one or 255 (it depends if binary image values are of type boolean with just 0 and 1
or 8-bit unsigned integer with values from 0 to 255). In case of this project
binary image is also inverted. In TEM images particles are black on white background
so in binary image particles pixels should be zero and background pixels should be one
and for this application it is better to invert it. There are several types of
thresholding techniques. Theese techniques uses histogram for calculating
appropriate threshold. There were used two different techniques in this project.
Minimum threshold method find pixel valu with minimum frequency in histogram and
this value is determined as threshold. Other method is called Otsu thresholding
and it is a bit more complicated algorithm. It consists in going throught every
pixel value and calculate variance in pixel values below this pixel and above this pixel
and finds pixel value with the smallest variance. This pixel value is determined as
threshold. Otsu thresholding is more advanced method and whould work better, but
in nanoparticles images with larger scale (there were just a few dart particles in center
of image) this method calculated too high threshold. More than just parctiles pixels
were determined as foreground. So it was chosen to use minimum thresholding method for
nanoparticles and Otsu thresholding method for nanorods.

After binarizing image there are used morphology operation to clean up image.
There are two basic operations called erosion and dilation. It works on principle
of interaction with structuring element which determines shape of pixel's neighborhood.
Dilation is used for filling holes and it increases size of objects. Algorithm changes pixel
value to one if there are enought white pixels in its neighborhood. Erosion is opposite
of dilation, it is used for removing small objects and it decreases size of objects.
Algorithm changes pixel value to zero if there are not enought white pixels in its
neighborhood. There are also two methods wchich consist of both erosion and dilation.
Opening first uses erosion and later dilation, it is used for removing small objects.
Closing first uses dilation and later erosion, it is used for removing small holes.
Opening and closing are better because it changes sizes of objects less than just erosion
and dilation. In scikit image library there are also fucntions designed perfectly
for removing small objects or holes. These scikit image function were also used
in this project for removing artifacts from background and fill holes in particles.

Main part of this project is image segmentation which separates particles so it could
be analyzed. There are plenty of methods used for segmentation and it is quite difficult
task to choose the best method and implement it. For this project watershed transformation
was chosen as main method. This algorithm starts with seeds and binary image. Seeds are
points or small clusters of points. In perferct world there should be one seed per
particle in image. Algorithm takes seeds and starts filling binary mask from theese points.
It is called watershed because seeds can be imagined as valleys (or local minima) and
algorithm fills every valley with water of different value starting from local minima.
It ends when it reaches edge of binary mask or when it meets water from another valley.
Labeled image is created, each pixel in one label has same number, background has zero.
There are more methods how to find seeds, in this project method called distance map
was used. It works on principle of finding for each zero pixel distance (number of pixels)
to the closest white pixel. Seeds were created by finding local maxima in distance fucntion.
It may happen that algorithm oversegments image so there are some small artifacts. So it
was decided to filter too small labels. It is done by algorithm calculates median of 
areas of labeles and going throught all labels and...

Watershed perfectly divides non-overlapping particles. In overlapping particles may occure
two cases, watershed divided particles well but did not count with their parts which
overlap, or it does not devide particles wll so it finds one big label. Due to this
problem it was decided to use also another segmentation method.
        \clearpage
        
        \section{Results}
        \pagestyle{plain}

\subsection{TEM images}\label{TEM_images}

TEM samples were prepared and TEM images were acquired. Various shapes and sizes of nanoparticles and nanorods were used. There were two samples of nanorods with unknown size and four sizes of nanoparticles with diameters 20 nm, 40 nm, 50 nm and 80 nm. There are examples of 20 nm nanoparticles (fig:\ref{fig:20nm}), 40 nm nanoparticles (fig:\ref{fig:40nm}) and nanorods (fig:\ref{fig:NRs}) with absorption peak on wavelength 800 nm.

\begin{figure}[h!]
\begin{center}
    \includegraphics[width=0.5\linewidth]{20nm.jpg}
    \caption{Raw TEM image of 20 nm nanoparticles}~\label{fig:20nm}
\end{center}
\end{figure}

\begin{figure}[h!]
\begin{center}
\begin{subfigure}(a)
    \includegraphics[width=0.4\linewidth]{40nm.jpg}
\end{subfigure}
\begin{subfigure}(b)
    \includegraphics[width=0.4\linewidth]{50nm.jpg}
\end{subfigure}
\caption{Raw TEM images of 40 nm (a) and 50 nm (b) nanoparticles}~\label{fig:40nm}
\end{center}
\end{figure}

\begin{figure}[h!]
\begin{center}
\begin{subfigure}(a)
    \includegraphics[width=0.4\linewidth]{NRs.jpg}
\end{subfigure}
\begin{subfigure}(b)
    \includegraphics[width=0.4\linewidth]{NRs2.jpg}
\end{subfigure}
\caption{Raw TEM images of nanorods}~\label{fig:NRs}
\end{center}
\end{figure}

\subsection{Python program}\label{program}

Python program was created (see Attachement). Program is able to process more images at once and it can run from command line with arguments, user may define folder with images or decide if results should be plotted or not. Program requires metadata in json format, which includes information about type of particles (nanoparticles or nanorods) and scale of TEM image. Program perform image segmentation and returns labeled images, average sizes and histograms of sizes.

\subsection{Program outputs}\label{outputs}

Program was tested with various input data, examples of results can be seen on figures \ref{fig:res20nm}, \ref{fig:res40nm} and \ref{fig:resNRs}.

\begin{figure}[h!]
\begin{center}
    \includegraphics[width=0.5\linewidth]{res20nm.png}
    \caption{Croped result labeled image of 20 nm nanoparticles}~\label{fig:res20nm}
\end{center}
\end{figure}

\begin{figure}[h!]
\begin{center}
\begin{subfigure}(a)
    \includegraphics[height=200px]{res40nm.jpg}
\end{subfigure}
\begin{subfigure}(b)
    \includegraphics[height=200px]{res50nm.png}
\end{subfigure}
    \caption{Croped result labeled images 40 nm (a) and 50 nm nanoparticles (b)}~\label{fig:res40nm}
\end{center}
\end{figure}

\begin{figure}[h!]
\begin{center}
\begin{subfigure}(a)
    \includegraphics[width=0.4\linewidth]{resNRs.jpg}
\end{subfigure}
\begin{subfigure}(b)
    \includegraphics[width=0.4\linewidth]{resNRs2.jpg}
\end{subfigure}
    \caption{Result labeled image on nanorods with absorption peak on 800 nm (a) and 660 nm (b)}~\label{fig:resNRs}
\end{center}
\end{figure}

Histograms of sizes were calculated and plotted, histogram of nanoparticles shows layout of diameters in input image (fig:\ref{fig:hist20nm}) and histogram of nanorods shows major and minor axis lengths in input image (fig:\ref{fig:histNRs}).

\begin{figure}[h!]
\begin{center}
    \includegraphics[width=0.8\linewidth]{hist20nm.jpg}
    \caption{Histogram of sizes of 20 nm nanoparticles}~\label{fig:hist20nm}
\end{center}
\end{figure}

\begin{figure}[h!]
\begin{center}
    \includegraphics[width=0.8\linewidth]{histNRs.jpg}
    \caption{Histogram of sizes of nanorods with absorption peak on 660 nm}~\label{fig:histNRs}
\end{center}
\end{figure}
        \clearpage
        
        \section{Discussion}
        \pagestyle{plain}

\subsection{Results evaluation}

Main result of this project is Python code for automatical analysis of TEM images of gold nanoparticles and nanorods. This code was debugged using input image \ref{fig:20nm}, because it is the most ideal image, there are advantages of this image: pretty good resolution, number of particles, fairly good contrast and homogeneous background. This image has also imperfections which can be used for solve these frequent issues, which are overlapping particles, hight resolution of image and small bright spots on particles. Overlapping particles were solved by using combination of two segmentation methods. Watershed transform was used for find all particles, but watershed divides ovelapping particles in the middle so it decreases its area, in some cases it even cannot divide some particles. So hough transform which finds specific shapes in image is used for detecting circles, this method is used just for areas with ovelapping particles. High resolution issue and bright spots are descibed in chapter \ref{image_errors}. Result of this sample image can be seen on figure \ref{fig:res20nm} and histogram of sizes of this image can be seen on figure \ref{fig:hist20nm} in chapter Results.

\subsection{Sample preparation errors}\label{measurment_errors}

The goals of the project were to prepare TEM samples on copper gird and acquire TEM images. Samples of nanoparticles and nanorods were prepared, but during TEM measurment there were another small particles on some samples. Theese small particles were DNA fragments and it means that some of grids must be used before and returned into box with new grids. This accident may cause errors during image segmentation while theese small particles are not removed well. This false detection of DNA particles can be seen on figure \ref{fig:res50nm}.

\begin{figure}[h!]
\begin{center}
    \includegraphics[width=0.5\linewidth]{res80nm.png}
    \caption{False detection of DNA particles in 80 nm nanoparticles TEM image}~\label{fig:res50nm}
\end{center}
\end{figure}

Another source of segmentation errors may be insufficiently dried samples. Samples of nanorods with absorption peak 800 nm can be seen on image \ref{fig:NRs}, this iimage has inhomogeneous background due to it was measured while sample was still a little wet. This is error which may happen more frequently so it is treated in software, however it declines quality of image. Wet sample can be seen on figure \ref{fig:wet}.

\begin{figure}[h!]
\begin{center}
    \includegraphics[width=0.5\linewidth]{wet.jpg}
    \caption{Inhomomogeneous background caused by wet sample}~\label{fig:wet}
\end{center}
\end{figure}

\subsection{TEM measurment errors}\label{image_errors}

Another sources of potential errors can by quality of TEM images. There are several issues which affects results of image analysis program. First issue is resolution. Microscope computer saves images in very high resolution 2044 x 2044 pixels and due to computational complexity of most functions, image resolution has to be reduced. For most of images rescaling does not matter, but there are images where particles are very tiny in compare with image size (example of this case can be seen on figure \ref{fig:40nm} (b) or figure \ref{fig:NRs} (b) in chapter Results). After rescaling theese particles have diameter for example 8 or 9 pixels, so accuracy of measurement extremely decrases. Solution for this problem could be manually cropping thees images before automatical analysis, but there is an issue. Images contain scale bar and it is located in corner of image and it is used to determine pixel size, so cropping image would remove this information. Appropriate solution for this problem was not found and it will be part of following project.

Another issue with TEM images quality can be contrast. Algorithm for particles analysis works on principle of substracting particles from background based on pixel intensity. Particles in some images contain stains with higher intensity which is similar or even higher than background intensity. So algorithm assigns these pixel to background not to particles. This issue is partly treated in code by using morfological operation for filling holes. This solution can solve smaller bright spots but it cannot do anything with cases where for example half of particle is bright. Some TEM images have low contrast overall so thresholding methods may not be so accurate it could be. Example of low contrast nanorods with even bright spots can be seen on \ref{fig:resNRs} (a) in chapter Results. In this image wet background is also problem, so it decreases quality of image even more. Bigger bright spots may cause oversegmentation of image. This case can be seen on figure \ref{fig:oversegment}.

\begin{figure}[h!]
\begin{center}
    \includegraphics[width=0.5\linewidth]{oversegment.png}
    \caption{Oversegmented nanorod}~\label{fig:oversegment}
\end{center}
\end{figure}

Last issue which is related to image quality and occured in sample images is badly focused image. It is problem especially in images with large number of small particles, because particles in samples are overlapping and in this case it is almost impossible to divide them well. Example of this case can be seen on figure \ref{fig:bad_focus}.

\begin{figure}[h!]
\begin{center}
    \includegraphics[width=0.5\linewidth]{discNRs.jpg}
    \caption{Bad focused nanorods}~\label{fig:bad_focus}
\end{center}
\end{figure}

In speech about overlapping particles there is also one issue with this which cannot be solved. Some samples contain places with a lot of particles laying on each other. In some extreme cases there is not even possible to see individual particles. This images obviously cannot be used for analysis. Program detects these huge clusters as one particle and it would bring enormous error in results so program have treated these cases removing such huge areas which were detected as one particle. Example of image with this case can be seen on figure \ref{fig:big_spot}.

\begin{figure}[h!]
\begin{center}
    \includegraphics[width=0.5\linewidth]{disc80nm.jpg}
    \caption{Image with huge cluster of particles}~\label{fig:big_spot}
\end{center}
\end{figure}

Last issue which was recorded is fact that microscope computer saves data into jpg format. It is understandable because jpg in compare with tiff has far lesser size, but compresses data and by compressing artifacts may occur. This can be problem in cases where original image have region with zero pixels and compressing causes that pixels with higher value (1, 2, 3 etc.) occur in this region, example of this region is scale bar. In image analysis program scale bar is detected by intensity threshold and artifacts may cause errors in this detection. It was solved by detecting more than just zero pixels and than taking just the biggest reion which is obviously scale bar.

\subsection{Calculations}

The main goals of this project is to create program which can automatically calculate sizes of nanoparticles and shapes of nanorods. Program returns histogram of diameters for nanoparticles and histogram of major and minor axis length for nanorods and also txt file with sizes of all particles, average sizes and for nanorods also aspect ratio. Errors in this section may be cause by all the cases described in chapter \ref{measurment_errors} and \ref{image_errors}. Detecting wrong particles may cause wrong average size, so particles which are too small or too big are removed. Some false particles may still not be removed due to many reasons so it may cause errors in results. Another source of results errors may be in resolution issue. If there is only a few pixels in one particle, accuracy of measurement extremely decreases. On figure \ref{fig:hist2} is example, where number of pixels in one particle was so small, that difference between particles were just in one pixel. This particles have diameter 8 and 9 pixels and when pixels are transformed into size in nanometers, it created this histogram.

\begin{figure}[h!]
\begin{center}
    \includegraphics[width=0.8\linewidth]{hist50nm.jpg}
    \caption{Histogram with just two sizes}~\label{fig:hist2}
\end{center}
\end{figure}
        \clearpage
        
        \section{Conclusion}
        \pagestyle{plain}

First goal of the project was to prepare TEM samples which was done using copper grid and potential errors caused by this part of project were evaluated in chapter \ref{measurment_errors} in Discussion. Five samples of nanoparticles of four different sizes were prepared and two samples of nanorods with different peak absorption wavelength were prepared.

Second goal of the project was to get TEM images of nanoparticles and nanorods. Several image of each sample were acquired by TEM by microscope technician. Examples of TEM images are showed in chapter \ref{TEM_images} in Results. Possible measurnemt errors were evaluated in chapter \ref{image_errors_errors} in Discussion.

Third goal of the project was to sugget methods for image segmentatin of these TEM images of nanoparticles and nanorods. Two segmentation methods were chosen on supervisor's recommendation. First method is called watershed transform and it was main method, other method is called hough transform and it was used just for overlapping particles. Both segmentation methods are descibed in chapter \ref{segmentation} in Methods.

Forth goal of the project was to create Python program for automatic analysis of nanoparticles and nanorods from TEM images. Code for this program can be seed in Attachement. Example outputs of program can be seen in chapter \ref{outputs} in Results and were evaluateed in Discussion. Program features were shortly descibed in chapter \ref{program} in Results.

Fifth goal of the project was to create GUI for program.
        \clearpage
    
    %-------------Literatura-------------------
    \clearpage	
    \renewcommand{\refname}{Reference} 	% Přejmenování Reference
    \printbibliography
    \clearpage
    
    %-------------Přílohy----------------------
        
    \end{document}