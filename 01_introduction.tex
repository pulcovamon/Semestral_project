\pagestyle{plain}

Nanoparticles and nanorods are widely used in biomedical research. For their usage it’s essential to know the size and shape of these particles in solution. The most used method for this estimation is transmission electron microscopy (TEM), which is very well known method, but data evaluation is very complicated and includes several image processing methods. The biggest part of image processing of TEM images is image segmentation. There are plenty of segmentation algorithms, but each data is unique and requires totally different approach. Other huge problem is that nanoparticles and nanorods are often overlapping, so the perfect method should divide them well and also take in account overlapping areas. TEM images also contain lot of noise, particles may have inhomogeneous background, even particles may not be of homogenous intensity. Simply there are so many sizes and shapes of nanoparticles or nanorods, that it is challenging to create algorithm which can automatically analyse TEM images of nanoparticles.